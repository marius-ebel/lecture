% !TEX root = ../../skript.tex

\chapter{Beispielkapitel}\label{chp:sample}
\minitoc

\section{Unterkapitel}

\subsection{Absatz}

\begin{hints}
	\item Pfeil-Aufzählungen
\end{hints}

\begin{hintbox*}[Exkurs: \ldots]
In \textbf{Exkursen} sind Erklärungen gegeben, die für das Verständnis eines Teilkapitels nützlich oder sogar erforderlich sind. Diese Erklärungen sind auf das Notwendigste reduziert.
\end{hintbox*}

\begin{hintbox}
	Hinweise oder Anmerkungen, die besondere Beachtung erfordern sind auf diese Weise hervorgehoben.
\end{hintbox}

\begin{enumerate}[label=\taskitemlabel,leftmargin=35pt]
\item Aufgaben, die entweder in der Vorlesung oder als Ergänzung außerhalb davon erledigt werden können, sind so dargestellt.
\end{enumerate}

\begin{lstlisting}
Code-Listings werden in nicht-proportionaler Schrift und 
mit Zeilennummern angegeben.
\end{lstlisting}
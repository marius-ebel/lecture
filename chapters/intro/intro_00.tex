% !TEX root = ../../skript.tex

\chapter{Einführung}

\section{Organisatorisches}

\subsection{Ablauf}

Die Termine für das Sommersemester 2019 sind folgendermaßen festgelegt:

\subsection{Ausgestaltung}

Um Ihren Lernerfolg zu erhöhen und den Weg zum Ziel (das Bestehen der Prüfung und darüber hinaus) einfacher zu gestalten, eine Ermutigung vorweg:


\subsection{Prüfungsanforderungen}

Die inhaltlichen Prüfungsanforderungen orientieren sich an den zuvor gegebenen Lernzielen. 

\subsection{Inhaltliche Anmerkungen}

Die Vielfalt an Werkzeugen, 

\subsection{Textsatz}

\begin{hintbox*}[Exkurs: \ldots]
In \textbf{Exkursen} sind Erklärungen gegeben, die für das Verständnis eines Teilkapitels nützlich oder sogar erforderlich sind. Diese Erklärungen sind auf das Notwendigste reduziert.
\end{hintbox*}

\begin{hintbox}
	Hinweise oder Anmerkungen, die besondere Beachtung erfordern sind auf diese Weise hervorgehoben.
\end{hintbox}

\begin{enumerate}[label=\taskitemlabel,leftmargin=35pt]
\item Aufgaben, die entweder in der Vorlesung oder als Ergänzung außerhalb davon erledigt werden können, sind so dargestellt.
\end{enumerate}

\begin{lstlisting}
Code-Listings werden in nicht-proportionaler Schrift und 
mit Zeilennummern angegeben.
\end{lstlisting}